\PassOptionsToPackage{unicode=true}{hyperref} % options for packages loaded elsewhere
\PassOptionsToPackage{hyphens}{url}
%
\documentclass[
]{article}
\usepackage{lmodern}
\usepackage{amssymb,amsmath}
\usepackage{ifxetex,ifluatex}
\ifnum 0\ifxetex 1\fi\ifluatex 1\fi=0 % if pdftex
  \usepackage[T1]{fontenc}
  \usepackage[utf8]{inputenc}
  \usepackage{textcomp} % provides euro and other symbols
\else % if luatex or xelatex
  \usepackage{unicode-math}
  \defaultfontfeatures{Scale=MatchLowercase}
  \defaultfontfeatures[\rmfamily]{Ligatures=TeX,Scale=1}
\fi
% use upquote if available, for straight quotes in verbatim environments
\IfFileExists{upquote.sty}{\usepackage{upquote}}{}
\IfFileExists{microtype.sty}{% use microtype if available
  \usepackage[]{microtype}
  \UseMicrotypeSet[protrusion]{basicmath} % disable protrusion for tt fonts
}{}
\makeatletter
\@ifundefined{KOMAClassName}{% if non-KOMA class
  \IfFileExists{parskip.sty}{%
    \usepackage{parskip}
  }{% else
    \setlength{\parindent}{0pt}
    \setlength{\parskip}{6pt plus 2pt minus 1pt}}
}{% if KOMA class
  \KOMAoptions{parskip=half}}
\makeatother
\usepackage{xcolor}
\IfFileExists{xurl.sty}{\usepackage{xurl}}{} % add URL line breaks if available
\IfFileExists{bookmark.sty}{\usepackage{bookmark}}{\usepackage{hyperref}}
\hypersetup{
  pdftitle={Turning your R scripts into reports},
  pdfauthor={David JPO'Sullivan},
  pdfborder={0 0 0},
  breaklinks=true}
\urlstyle{same}  % don't use monospace font for urls
\usepackage[margin=1in]{geometry}
\usepackage{color}
\usepackage{fancyvrb}
\newcommand{\VerbBar}{|}
\newcommand{\VERB}{\Verb[commandchars=\\\{\}]}
\DefineVerbatimEnvironment{Highlighting}{Verbatim}{commandchars=\\\{\}}
% Add ',fontsize=\small' for more characters per line
\usepackage{framed}
\definecolor{shadecolor}{RGB}{248,248,248}
\newenvironment{Shaded}{\begin{snugshade}}{\end{snugshade}}
\newcommand{\AlertTok}[1]{\textcolor[rgb]{0.94,0.16,0.16}{#1}}
\newcommand{\AnnotationTok}[1]{\textcolor[rgb]{0.56,0.35,0.01}{\textbf{\textit{#1}}}}
\newcommand{\AttributeTok}[1]{\textcolor[rgb]{0.77,0.63,0.00}{#1}}
\newcommand{\BaseNTok}[1]{\textcolor[rgb]{0.00,0.00,0.81}{#1}}
\newcommand{\BuiltInTok}[1]{#1}
\newcommand{\CharTok}[1]{\textcolor[rgb]{0.31,0.60,0.02}{#1}}
\newcommand{\CommentTok}[1]{\textcolor[rgb]{0.56,0.35,0.01}{\textit{#1}}}
\newcommand{\CommentVarTok}[1]{\textcolor[rgb]{0.56,0.35,0.01}{\textbf{\textit{#1}}}}
\newcommand{\ConstantTok}[1]{\textcolor[rgb]{0.00,0.00,0.00}{#1}}
\newcommand{\ControlFlowTok}[1]{\textcolor[rgb]{0.13,0.29,0.53}{\textbf{#1}}}
\newcommand{\DataTypeTok}[1]{\textcolor[rgb]{0.13,0.29,0.53}{#1}}
\newcommand{\DecValTok}[1]{\textcolor[rgb]{0.00,0.00,0.81}{#1}}
\newcommand{\DocumentationTok}[1]{\textcolor[rgb]{0.56,0.35,0.01}{\textbf{\textit{#1}}}}
\newcommand{\ErrorTok}[1]{\textcolor[rgb]{0.64,0.00,0.00}{\textbf{#1}}}
\newcommand{\ExtensionTok}[1]{#1}
\newcommand{\FloatTok}[1]{\textcolor[rgb]{0.00,0.00,0.81}{#1}}
\newcommand{\FunctionTok}[1]{\textcolor[rgb]{0.00,0.00,0.00}{#1}}
\newcommand{\ImportTok}[1]{#1}
\newcommand{\InformationTok}[1]{\textcolor[rgb]{0.56,0.35,0.01}{\textbf{\textit{#1}}}}
\newcommand{\KeywordTok}[1]{\textcolor[rgb]{0.13,0.29,0.53}{\textbf{#1}}}
\newcommand{\NormalTok}[1]{#1}
\newcommand{\OperatorTok}[1]{\textcolor[rgb]{0.81,0.36,0.00}{\textbf{#1}}}
\newcommand{\OtherTok}[1]{\textcolor[rgb]{0.56,0.35,0.01}{#1}}
\newcommand{\PreprocessorTok}[1]{\textcolor[rgb]{0.56,0.35,0.01}{\textit{#1}}}
\newcommand{\RegionMarkerTok}[1]{#1}
\newcommand{\SpecialCharTok}[1]{\textcolor[rgb]{0.00,0.00,0.00}{#1}}
\newcommand{\SpecialStringTok}[1]{\textcolor[rgb]{0.31,0.60,0.02}{#1}}
\newcommand{\StringTok}[1]{\textcolor[rgb]{0.31,0.60,0.02}{#1}}
\newcommand{\VariableTok}[1]{\textcolor[rgb]{0.00,0.00,0.00}{#1}}
\newcommand{\VerbatimStringTok}[1]{\textcolor[rgb]{0.31,0.60,0.02}{#1}}
\newcommand{\WarningTok}[1]{\textcolor[rgb]{0.56,0.35,0.01}{\textbf{\textit{#1}}}}
\usepackage{graphicx,grffile}
\makeatletter
\def\maxwidth{\ifdim\Gin@nat@width>\linewidth\linewidth\else\Gin@nat@width\fi}
\def\maxheight{\ifdim\Gin@nat@height>\textheight\textheight\else\Gin@nat@height\fi}
\makeatother
% Scale images if necessary, so that they will not overflow the page
% margins by default, and it is still possible to overwrite the defaults
% using explicit options in \includegraphics[width, height, ...]{}
\setkeys{Gin}{width=\maxwidth,height=\maxheight,keepaspectratio}
\setlength{\emergencystretch}{3em}  % prevent overfull lines
\providecommand{\tightlist}{%
  \setlength{\itemsep}{0pt}\setlength{\parskip}{0pt}}
\setcounter{secnumdepth}{-2}
% Redefines (sub)paragraphs to behave more like sections
\ifx\paragraph\undefined\else
  \let\oldparagraph\paragraph
  \renewcommand{\paragraph}[1]{\oldparagraph{#1}\mbox{}}
\fi
\ifx\subparagraph\undefined\else
  \let\oldsubparagraph\subparagraph
  \renewcommand{\subparagraph}[1]{\oldsubparagraph{#1}\mbox{}}
\fi

% set default figure placement to htbp
\makeatletter
\def\fps@figure{htbp}
\makeatother


\title{Turning your R scripts into reports}
\author{David JPO'Sullivan}
\date{2/25/2020}

\begin{document}
\maketitle

\hypertarget{what-is-r-markdown}{%
\subsection{What is R Markdown?}\label{what-is-r-markdown}}

This is an R Markdown document. Markdown is a simple formatting syntax
for authoring HTML, PDF, and MS Word documents. For more details on
using R Markdown, see the following websites
\href{http://rmarkdown.rstudio.com}{here} and
\href{https://rmarkdown.rstudio.com/articles_intro.html}{here}. R
Markdown is a powerful tool for automating your report creation process.
When you click the \textbf{Knit} button, a document will be generated
that includes both content as well as the output of any embedded R code
chunks within the document. You can embed an R code chunk like this.

\begin{Shaded}
\begin{Highlighting}[]
\CommentTok{# readin the data}
\KeywordTok{glimpse}\NormalTok{(credit_slr_df)}
\end{Highlighting}
\end{Shaded}

\begin{verbatim}
## Observations: 226
## Variables: 2
## $ bill_amt1 <dbl> 119287, 4670, 12547, 277822, 59143, 874, 21854, 41906, 57...
## $ bill_amt2 <dbl> 116995, 4670, 14699, 255167, 58612, -256, 17376, 17969, 2...
\end{verbatim}

These chunks run segments of code from the analysis. They can be printed
in the resulting document or not, but the information they produce is
still available for analysis. In the following section we are going to
turn the R script that we used to perform the hypothesis tests and
estimation of the linear regression model into an automatically
generated report.

\newpage

\hypertarget{credit-modelling}{%
\section{Credit modelling}\label{credit-modelling}}

\hypertarget{hypothesis-testing-on-credit-data}{%
\subsection{Hypothesis testing on credit
data}\label{hypothesis-testing-on-credit-data}}

\hypertarget{difference-in-population-means}{%
\subsubsection{Difference in population
means}\label{difference-in-population-means}}

We are interested in the differences in the population means of
\texttt{total\_bill\_amt} between those that \emph{defaulted} and those
that do not. To answer this question, we use a 5\% level of significance
(\(\alpha = 0.05\)).

The null and alternative hypothesis for the t-test are

\begin{itemize}
\tightlist
\item
  \(H_0:\) There is no difference between the population means
  (\(\mu_1 = \mu_2\)).
\item
  \(H_A:\) There is a difference between the population means
  (\(\mu_1 \ne \mu_2\)).
\end{itemize}

The resulting \(p\)-value from the t-test is \(0.03\), which is less
than \(\alpha = 0.05\) than we reject the null hypothesis. The
population means are different between the default and non-default
group.

Additionally, we can examine the confidence interval for the mean
difference between the two groups. The 95\% CI for the mean difference
between the groups is \([754, \ensuremath{1.4853\times 10^{4}}]\).
There, we are 95\% certain the non-default group owes, on average,
between \(754\) and \(\ensuremath{1.4853\times 10^{4}}\) more than the
default group.

\hypertarget{paired-sample-t-test}{%
\subsubsection{Paired sample t-test}\label{paired-sample-t-test}}

\textbf{Insert markdown text and/or code here to:}

\begin{itemize}
\tightlist
\item
  Do a paired sample t-test between \texttt{bill\_amt1} and
  \texttt{bill\_amt2}.
\item
  Comment on the set up of the test and result using the steps outlined
  in the lecture notes.
\item
  Calculate the confidence interval for the difference and comment
\end{itemize}

\hypertarget{modelling-customer-spending-behaviour-between-months}{%
\subsection{Modelling customer spending behaviour between
months}\label{modelling-customer-spending-behaviour-between-months}}

Here we will investigate if we can predict \texttt{bill\_amt1} using
\texttt{bill\_amt2}. The Pearson's correlation coefficient between the
two variables is 0.97, which indicate that there is a very strong linear
relationship. To confirm this, we visually inspect a scatter plot of the
two variables. From the following graph, we note that there is a linear
trend between \texttt{bill\_amt1} and \texttt{bill\_amt2}.

\begin{verbatim}
## `geom_smooth()` using method = 'loess' and formula 'y ~ x'
\end{verbatim}

\begin{figure}
\centering
\includegraphics{Turing_R_into_reports_with_Rmarkdown_start_files/figure-latex/slr-scatter-1.pdf}
\caption{Scatter plot of \texttt{bill\_amt1} and \texttt{bill\_amt2}}
\end{figure}

The line of best fit is estimated as:

\[ y = -1838.74 + 1.07\times x \]

Examining the confidence intervals for the slope. We are 95\% certain
that the true slope (\(b_1\)) for \texttt{bill\_amt2} is in the range
(1.03, 1.1).

\textbf{Insert markdown text and/or code here to:}

\begin{enumerate}
\def\labelenumi{\arabic{enumi}.}
\tightlist
\item
  Comment on the confidenc interval.
\item
  Interpret the slope.
\item
  State the \(R^2\) and commonent on the goodness of fit
\end{enumerate}

\newpage

\hypertarget{model-diagnostics}{%
\subsubsection{Model diagnostics}\label{model-diagnostics}}

We can examine the quality of the fits and check that the model
statistics the underlying assumptions of the model using the following
diagnostic plot of the residuals.

\begin{figure}
\centering
\includegraphics{Turing_R_into_reports_with_Rmarkdown_start_files/figure-latex/slr-diag-1.pdf}
\caption{Diagnostic plot for regression.`}
\end{figure}

There is some evidence of departures from normality in the tails of the
Q-Q plot (but the values between -2 to 2 fit reasonably well). We also
have evidence of a point of high leverage (sample number 119), might be
work to see what is so special about that person.

It seems that it is maybe reasonable to use this model to predict the
\texttt{bill\_amt1} using \texttt{bill\_amt2}. But we should expect the
model to be inaccurate for very large or very small values.

\newpage

\hypertarget{accuracy-of-predictions}{%
\subsection{Accuracy of predictions}\label{accuracy-of-predictions}}

\textbf{Insert markdown text and/or code here to:}

\begin{enumerate}
\def\labelenumi{\arabic{enumi}.}
\tightlist
\item
  Visually inspect the a plot of the fitted linear regression model, the
  fitted data and the test data.\\
\item
  Comment on why the model may not predict as well on the test data than
  the training data.
\end{enumerate}

\end{document}
